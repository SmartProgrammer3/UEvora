\documentclass{report}
\usepackage[portuguese]{babel}
\usepackage[utf8]{inputenc}
\usepackage{amsmath}
\usepackage{natbib}
\usepackage[nottoc]{tocbibind}
\usepackage{graphicx}
\usepackage{subcaption}
\usepackage{geometry}

\title{\textbf{Conversor de Imagem \\
{\Large RGB-Gray}} \\
\vspace{1.5cm}
\includegraphics[width=0.4\textwidth]{Imagem1.png}
}
\author{Pedro Emílio nº 52649 \\ Luís Carvalho nº 51817}
\vspace{3cm}
\date{\textbf{Departamento de Informática \\ Arquitetura de computadores I \\ Docente: Miguel Barão}}

\geometry {
    top=.75cm,
    bottom=.75cm,
    rmargin=2cm,
    lmargin=2cm,
}
\begin{document}

\maketitle

\tableofcontents

\chapter{Introdução}
Este projeto, da disciplina de Arquitetura de Computadores I, divide-se em várias partes.\\
Na primeira parte pretende-se desenvolver um conjunto de funções, na linguagem assembly RISC-V, que façam uma conversão de uma imagem jpg, a Figura 1.1, para uma imagem em tons de Gray, a Figura 1.2.\\
De seguida pretende-se filtrar a imagem através do operador de Sobel vertical e horizontal.\\
Visto que foram obtidas as imagens filtradas, através do operador de Sobel vertical e horizontal (Figura 1.3 e Figura 1.4), foi criado uma função que combina essas duas imagens obtendo-se assim a imagem com o fundo branco e contornos pretos, a Figura 1.5.\\

\begin{figure}[h]
    \centering
    
    \begin{minipage}{.49\textwidth}
    \centering
    \includegraphics[width=.6\textwidth]{Lena.jpg}
    \caption{Lena}
    \label{fig:my_label}
    \end{minipage}
    \hfill
    \begin{minipage}{.49\textwidth}
    \centering
    \includegraphics[width=.6\textwidth]{LenaGray.jpg}
    \caption{Imagem convertida \\ para escala de Gray.}
    \label{fig:my_label}
    \end{minipage}
\end{figure}

\begin{figure}
    \centering
    \includegraphics[width=.3\textwidth]{LenaSobelVertical.jpg}
    \caption{Imagem filtrada pelo \\ operador Sobel vertical.}
    \label{fig:my_label}
\end{figure}

\begin{figure}[h]
    \centering
    
    \begin{minipage}{.49\textwidth}
    \centering
    \includegraphics[width=.6\textwidth]{LenaSobelHorizontal.jpg}
    \caption{Imagem filtrada pelo \\ operador Sobel horizontal..}
    \label{fig:my_label}
    \end{minipage}
    \hfill
    \begin{minipage}{.49\textwidth}
    \centering
    \includegraphics[width=.6\textwidth]{imagemContorno.jpg}
    \caption{Contornos obtidos combinando os resultados \\ dos operadores Sobel horizontal e vertical.}
    \label{fig:my_label}
    \end{minipage}
\end{figure}

\chapter{Funções Principais}

\section{read\_rgb\_image}
Esta função tem o propósito de ler um ficheiro com uma imagem no formato RGB para um array em
memória e tem como parâmetros uma string com o nome do ficheiro que lerá e
o endereço de um buffer onde a imagem deverá ser escrita.

\section{write\_to\_gray}
O objetivo desta função é escrever uma imagem em formato GRAY num ficheiro e 
tem como parâmetros o nome de um ficheiro, um buffer com a imagem e o cumprimento do mesmo.

\section{rgb\_to\_gray}
Esta função tem como propósito converter uma imagem a cores RGB para uma imagem em tons de GRAY e tem como parâmetros um buffer com a imagem RGB, um buffer onde será colocada a imagem em formato GRAY e o tamanho da imagem.

\section{convolution}
Esta função tem como objetivo calcular a convolução de uma imagem A com um operador Sobel (matriz
3×3) e por sua vez colocar o resultado numa matriz B. Esta função tem como parâmetros um buffer
com a matriz A, um buffer com um dos operadores Sobel e um buffer que conterá
a imagem filtrada B. A função pode assumir internamente uma dimensão de imagem
fixa ou ter a dimensão como parâmetros.

\section{contour}
O propósito desta função é calcular a imagem final (a imagem com o fundo branco e contornos pretos) combinando as duas imagens convolvidas e tem como parâmetros dois buffers com as imagens a combinar e um buffer que vai conter o resultado final.

\chapter{Funções Auxiliares}

\section{write\_sobelHorizontal\_image}
Esta função escreve num ficheiro gerado (.gray), a imagem com a convulação na horizontal, ou seja, a imagem  filtrada pelo operador Sobel horizontal.

\section{write\_sobelVertical\_image}
Esta função escreve num ficheiro gerado (.gray), a imagem com a convulação na vertical, ou seja, a imagem filtrada pelo operador Sobel vertical.

\section{write\_imagemContorno}
Esta função escreve num ficheiro gerado (.gray), a imagem final, com os contornos obtidos combinando os 
resultados dos operadores Sobel horizontal e vertical,

\chapter{Conclusão}
Como foi referido anteriormente na introdução, o objetivo deste projeto era criar um programa em assembly RISC-V capaz de converter uma imagem jpg numa imagem de fundo branco com contornos pretos.\\
Na realização do projeto foi possível atingir todos os objetivos apresentados pelo docente da disciplina, apesar de ter sido muito difícil conciliar o tempo para realizar o projeto enquanto se estudava para exames de outras disciplinas.\\


\end{document}
