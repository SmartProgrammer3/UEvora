\documentclass{article}
\usepackage[utf8]{inputenc}

\title{Citações e Normas Bibliográficas}
\author{Afonso Hipólito [48781], Luís Carvalho[51817]}
\date{8 de março de 2023}

\usepackage[sorting=none]{biblatex}
\bibliography{references}
%plain text
\bibliographystyle{plain}
\PassOptionsToPackage{hyphens}{url}\usepackage{url}

%evita overflow
\setcounter{biburllcpenalty}{7000}
\setcounter{biburlucpenalty}{8000}

\begin{document}
\maketitle

\section{Introdução}
Neste trabalho utilizámos, como é pedido no enunciado, o \textbf{bibtex} para fazermos as referênias aos artigos mais relevantes que pesquisámos no trabalho 2, da pesquisa bibliográfica. 

\section{Citações}

    % Google Scholar - "bias" + "variance"
    \cite{citeseerx} \\
    \cite{yaroslavvb} \\
    % Yahoo - "Overfitting"
    \cite{ibmcloud} \\
    \cite{elitedatascience} \\
    \cite{geeksover} \\
    % Yahoo - "bias" + "variance"
    \cite{medium} \\
    \cite{medium2} \\ 
    \cite{analyticsvidhya} \\
    % Yahoo! - "overfitting" + "bias" + "variance"
    \cite{towardsdatascience} \\
    \cite{dustinstansbury}
    

\printbibliography[title=Referências]

\end{document}
