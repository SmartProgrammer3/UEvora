\documentclass{article}
\usepackage[utf8]{inputenc}

\title{Robôs Domésticos vs Robôs Industriais}
\author{Afonso Hipólito[48781], Luís Carvalho[51817]}
\date{18 de Março  2023}

\usepackage[sorting=none]{biblatex}
\bibliography{references}

\PassOptionsToPackage{hyphens}{url}\usepackage{url}

\setcounter{biburllcpenalty}{7000}
\setcounter{biburlucpenalty}{8000}

\begin{document}

\maketitle

\section{Resumo}
O tema que pretendemos abordar é sobre os rôbos domésticos e industrais, visto que nos últimos anos, todos eles têm tido uma enorme evolução, apesar dessa evolução passar despercebida aos olhos de todos nós.
\\
\\
Pretendemos explicar o que são os \textbf{robôs domésticos} \cite{whatare}\cite{DomesticRobots}, o que são os \textbf{robôs industriais} \cite{robotnik_2022}, e a sua respetiva \textbf{evolução e história} \cite{wevolver}\cite{history}. O nosso objetivo é também expor algumas \textbf{vantagens   e desvantagens} \cite{advantages}\cite{stevens_2022} para cada um destes tipos de robôs, quer no nosso dia a dia, quer a um nível mais generalizado.
\\
\\
Duas das razões para termos escolhido este tema, para além de estar relacionado com a nossa àrea e de ser interessante, é o facto da robótica estar relacionada com a \textbf{inteligência artificial} \cite{martin_2022}\cite{helfrich_2022} e a sua influência que tem vindo a ter no \textbf{mercado financeiro} \cite{market}. Além disso, é um campo que está em constante crescimento e evolução, e as possibilidades práticas são imensas.
\\
\\
Por úlimo, planeamos apresentar alguns dos exemplos mais relevantes quer para os robôs domésticos, quer para os industriais (em fotos), bem como perceber as suas funções, as capacidades/características de cada um e ficar a conhecer um pouco sobre o seu funcionamento.


\printbibliography[title=Referências]


\end{document}
